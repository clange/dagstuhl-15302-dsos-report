%%%%%%%%%%%%%%%%%%%%%%%%%%%%%%%%%%%%%%%%%%%%%%%%%%%%%%%%%%%%%%%%%%%%%%%%%%%%%%%
%%%%%%%%%%%%%%%%%%%%%%%%%%%%%%%%%%%%%%%%%%%%%%%%%%%%%%%%%%%%%%%%%%%%%%%%%%%%%%%
% %
% Title: Automatically generated Template for Dagstuhl Reports %
% Script used: abstracts-listing_tex.wml %
% originally developped by Tobias Maurer %
% 2005-02-15: Layout Template (by Jutta Huhse) %
% 2011-02-28: Layout adapted to Dagstuhl Reports (by Marc Herbstritt) %
% %
%%%%%%%%%%%%%%%%%%%%%%%%%%%%%%%%%%%%%%%%%%%%%%%%%%%%%%%%%%%%%%%%%%%%%%%%%%%%%%%
% 
% This document requires the Dagstuhl Reports LaTeX style package dagrep.cls 
% and corresponding graphic files. 
% It can be downloaded from the following URL:
%
% http://www.dagstuhl.de/???
%
%%%%%%%%%%%%%%%%%%%%%%%%%%% head declarations %%%%%%%%%%%%%%%%%%%%%%%%%%%%%%%%%
%%%%%%%%%%%%%%%%%%%%%%%%%%%%%%%%%%%%%%%%%%%%%%%%%%%%%%%%%%%%%%%%%%%%%%%%%%%%%%%


%This is a template for producing reports for "Dagstuhl Reports".
%See dagrep.pdf for further information.

\documentclass[a4paper,USenglish]{dagrep}
  %for A4 paper format use option "a4paper", for US-letter use option "letterpaper"
  %for british hyphenation rules use option "UKenglish", for american hyphenation rules use option "USenglish"
  %for section-numbered lemmas etc., use "numberwithinsect"

\usepackage{microtype}%if unwanted, comment out or use option "draft"
\usepackage[babel]{csquotes}
\MakeAutoQuote{“}{”}

\bibliographystyle{plain}%the recommended bibstyle

% Preamble with header information 
\subject{Report from Dagstuhl Seminar 15302}
\title{Perspectives Workshop: Digital Scholarship and Open Science in Psychology and the Behavioral Sciences}
\titlerunning{15302 -- Perspectives Workshop: Digital Scholarship and Open Science in Psychology and the Behavioral Sciences}%optional

\author[1]{Alexander Garcia Castro}
\author[2]{Janna Hastings}
\author[3]{Robert Stevens}
\author[4]{Erich Weichselgartner}
\authorrunning{Alexander Garcia Castro, Janna Hastings, Robert Stevens, and Erich Weichselgartner} %optional
\affil[1]{Technical University of Madrid, ES, \texttt{alexgarciac@gmail.com}}
\affil[2]{European Bioinformatics Institute – Cambridge, GB, \texttt{hastings@ebi.ac.uk}}
\affil[3]{University of Manchester, GB, \texttt{robert.stevens@manchester.ac.uk}}
\affil[4]{ZPID – Trier, DE, \texttt{wga@zpid.de}}


%Organizer macros:%%%%%%%%%%%%%%%%%%%%%%%%%%%%%%%%%%%%%%%%%%%%%%%%%%%%%
\seminarnumber{15302}
\semdata{\emph{19}.--\emph{24}.~\emph{July}, \emph{2015} -- \url{http://www.dagstuhl.de/15302}}
\subjclass{J.4 Social and Behavioral Sciences, I.7.4 Electronic Publishing, H.3.5 Online Information Services, H.3.7 Digital Libraries} %mandatory
\keywords{Digital Scholarship, Open Science, Psychology, Behavioral Sciences, e-Science} %mandatory
\additionaleditors{Christoph Lange} %optional
%%%%%%%%%%%%%%%%%%%%%%%%%%%%%%%%%%%%%%%%%%%%%%%%%%%%%%%%%%%%%%%%%%%%%%%

%Dagstuhl editorial office macros:%%%%%%%%%%%%%%%%%%%%%%%%%%%%%%%%%%%%%
\volumeinfo%(easychair interface)
  {Alexander Garcia Castro, Janna Hastings, Robert Stevens, and Erich Weichselgartner}%editor names
  {5}%number of editors
  {Perspectives Workshop: Digital Scholarship and Open Science in Psychology and the Behavioral Sciences}%seminar title
  {1}%volume
  {1}%issue
  {1}%starting page number
\DOI{10.4230/DagRep.1.1.1}%(DagRep.<volume no>.<issue no>.<firstpage>)
%%%%%%%%%%%%%%%%%%%%%%%%%%%%%%%%%%%%%%%%%%%%%%%%%%%%%%%%%%%%%%%%%%%%%%%

\usepackage{todonotes}

\frenchspacing

%%%%%%%%%%%%%%%%%%%%%%%%%%%%%%%%%%%%%%%%%%%%%%%%%%%%%%%%%%%%%%%%%%%%%%%%%%%%%%%
%%%%%%%%%%%%%%%%%%%%%%%% begin of document %%%%%%%%%%%%%%%%%%%%%%%%%%%%%%%%%%%%
%%%%%%%%%%%%%%%%%%%%%%%%%%%%%%%%%%%%%%%%%%%%%%%%%%%%%%%%%%%%%%%%%%%%%%%%%%%%%%%
\begin{document}

\maketitle

%------------------------------------------------------------------------
%------------------------------------------------------------------------
% this is a standard text with some seminar specific information
\begin{abstract}
This report documents the program and the outcomes of Dagstuhl Seminar 15302 ``Perspectives Workshop: Digital Scholarship and Open Science in Psychology and the Behavioral Sciences''.
This workshop addressed the problem of facilitating the construction of an integrative digital scholarship and open science infrastructure in psychology and the behavioral sciences by utilizing the Web an integrative platform for e-Science.
A particular focus was on sharing research data and experiments to improve reproducibility.
The participants presented first steps in this direction in their communities, and worked out an initial action plan for establishing digital scholarship and open science more broadly.
\end{abstract}
%------------------------------------------------------------------------

\section{Executive Summary}

\summaryauthor[ Garcia Castro, Alexander; Hastings, Janna; Lange, Christoph; Stevens, Robert; Weichselgartner, Erich ]{Garcia Castro, Alexander; Hastings, Janna; Lange, Christoph; Stevens, Robert; Weichselgartner, Erich}
\license
Researchers across many domains have invested significant resources to improve
transparency, reproducibility, discoverability and, in general, the ability to
share and empower the community.  Digital Scholarship and Open Science are
umbrella terms for the movement to make scientific research, data and
dissemination accessible to all members of an inquiring society, amateur or
professional. Digital infrastructures are an essential prerequisite for such
open science and digital scholarship; the biomedical domain illustrates this
symbiosis. An impressive digital infrastructure has been built; this allows us
to correlate information from genomes to diseases, and, by doing so, to support
personalized medicine. A high degree of interdisciplinary work was necessary in
building this infrastructure; the large quantities of data being produced, the
high degree of interrelatedness, and, most of all, the need for this mingling of
many types of data in a variety of forms forged this collaboration across the
community and beyond.

The Behavioral Sciences, comprising psychology but also psychobiology,
criminology, cognitive science and neuroscience, are also producing data at
significant rates; by the same token, understanding mental health disorders
requires correlating information from diverse sources – e.g. cross-referencing
clinical, psychological, and genotypic sources. For example, flagship projects
such as the Brain Activity Map (BAM, also known as the BRAIN
initiative\footnote{\url{http://www.braininitiative.nih.gov/}}) are generating
massive amounts of data with potential benefit to mental health and psychology;
conversely, projects like BAM could benefit from information currently being
generated by psychologists. Our ability to make continued progress in
understanding the mind and brain depends on finding new ways to organize and
synthesize an ever-expanding body of information.

The \emph{`Digital Scholarship and Open Science in Psychology and the Behavioral
Sciences'} Dagstuhl Perspectives Workshop was conceived with one problem in
mind: that of facilitating the construction of an integrative infrastructure in
Psychology and Behavioral Sciences. The motivation for this workshop was to
\emph{`foster the discussion around the problem of understanding the Web as an
integrative platform, and how e-science can help us to do better research.'}
With these points in mind, we gathered an interdisciplinary group of experts,
including computer scientists, psychologists and behavioral scientists. In their
research, they are addressing issues in data standards, e-science, ontologies,
knowledge management, text mining, scholarly communication, semantic web,
cognitive sciences, neurosciences, and psychology. Throughout the Workshop, this
group worked on devising a roadmap for building such an interoperability layer. 

The seminar started with a number of keynote sessions from well-known
authorities in each area to introduce the necessary background and form a common
baseline for later discussions. A core theme that emerged was the cross-domain
challenge in establishing a common language.  We jointly undertook the effort to
define an integrative scenario illustrating how digital infrastructures could
help psychologists and behavioral scientists to do research that takes advantage
of the new digital research landscape. In order to achieve this, the
computational scientists needed to better understand the current working
practices of the psychologists. For instance, the nature and structure of their
data and experiments; moreover, computer scientists needed to understand the
flow of information, from the conception of an idea, through defining a study
plan, executing it and finally having the investigation published. They learned
that the work of psychologists and behavioral scientists strongly relies on
questionnaires and experiments as ways of collecting data, and on statistics as
a tool for analyzing data, and that the replicability of experiments is a key
concern. In a similar vein, psychologists and behavioral scientists needed
concrete examples illustrating how computer science enables FAIR (= findable,
accessible, interoperable and reusable) infrastructures that allow researchers
to discover and share knowledge – bearing in mind data protection issues. 

Two break-out groups were organized. The purpose was to have a full picture of
digital scholarship in action when applied to psychology and behavioral
investigations, most importantly e-science assisting researchers in sharing,
discovering, planning and running investigations. The full research life cycle
had to be considered. Both groups worked up their respective scenarios
independently. The visions were then exchanged in an inter-group meeting.
Interestingly, various issues arose when discussing the specifics from each
vision for digital scholarship; for instance, the importance of understanding
scholarly communication beyond the simple act of getting one's results
published. Furthermore, the need to integrate tools into platforms where
researchers could openly register their projects and plan and manage their
workflows, data, code and research objects, was extensively discussed. Within
this framework, the need for controlled vocabularies, standards for publishing
and documenting data and metadata, persistent identifiers for datasets, research
objects, documents, organizations, concepts and people, open APIs to existing
services and instruments, and reporting structures were understood; these
elements were articulated in the examples where the researchers and research
were at the center of the system. Discussions also addressed fears in the
community and thus the need to open up the current research landscape in small
steps.

The seminar proved to be a fertile discussion field for interdisciplinary
collaborations and research projects across previously disparate fields with the
potential of significant impact in both areas. The need for a digital
infrastructure in psychology and behavioral sciences was accepted by all the
attendants; communicating this message with a clear implementation vision to
funding agencies, professional societies and the community in general was
identified as a key priority. It was decided that we needed another meeting in
2016; during that follow-up, the emphasis should be on developing a research
agenda. As this is a relatively new topic in psychology and behavioral sciences,
it was also decided to contact publishers and professional organizations, e.g.
the Sloan Foundation, the APA and the APS, and work with them in conveying the
message about increasing openness. If we want to understand how cognition is
related to the genome, proteome and the dynamics of the brain, then
interoperability, data standards and digital scholarship have to become a common
purpose for this community. Funding has to be made available, initially for an
assessment of the uptake of existing key resources and infrastructures, and then
for implementing further Digital Scholarship and Open Science infrastructures as
well as for building the skills in a community that is not yet widely familiar
with the relevant enabling technologies. Finally, once sufficient technical
support is in place, sustainable incentives for sharing research objects should
be put in practice.

\tableofcontents

\section{Overview of Talks}


%-------------------------------------------------------
%\newpage
\abstracttitle{Mental illnesses, knowledge representation and data sharing}
\abstractauthor[ Xavier Aime ]{Xavier Aime (LIMICS INSERM U 1142 – Paris, FR)}
\license
\jointwork{Aim\'{e}, Xavier; Richard, Marion; Charlet, Jean; Krebs, Marie-Odile}
\abstractref{Richard, M.; Aimé, X.; Krebs, M. \& Charlet, J. Enrich classifications in psychiatry with textual data: an ontology for psychiatry including social concepts Studies in Health Technology and Informatics, 2015, 210, 221-223}
%\abstractrefurl{} % preferrably DOI-based URL
Mental illnesses have a major impact on public health but their pathophysiology
remains largely unknown and their treatments insufficient. One of the main
difficulties is that these disorders are defined only on the basis of clinical
syndromes issued from historical descriptions, now with consensual criteria in
the international classifications (CIM 10 from WHO or DSM-IV from American
Psychiatric Association). An additional hallmark of psychiatry is the apparent
lack of specificity of the biological markers or risk factors, when identified,
and the overlap of symptoms across diagnostic categories. There is thus an
urgent need to be able to define more precisely the phenotype, or profile of
anomalies, at the individual level, by taking into account numerous and
heterogeneous ways of characterization, collected in large clinical databases.
The domain of psychological disorders raises several challenges that need to be
addressed, as large amount and diversity of sources and nature of information,
the evolutivity of symptoms and diachronic trajectories of mental disorders
across a person's life span, and special requirements of all human rights
documents and protection of privacy. Research in this area is data intensive,
which means that data sets are large and highly heterogeneous; therefore the use
of inappropriate models would lead to inappropriate (if not flawed) results.
Clinical data is complex, non trivial, and redundant. To create knowledge from
such data, researchers must integrate and share these large and diverse data
sets. This presents daunting computer science challenges such as representation
of data that is suitable for computational inference (knowledge representation
with an ontology such as OntoPsychia \cite{Richard}), and linking
heterogeneous data sets (data integration – unfortunately, data integration and
sharing are hampered by legitimate and widespread privacy concerns). The use of
an ontology, associated with dedicated tools, will allow also (1) to perform
semantic research in Patient Discharges Summary (PDS), (2) to represent the
comorbidity, (3) to index PDS for the constitution of cohorts, and (4) to
identify resistant patient's profiles.

\begin{thebibliography}{0}
\bibitem{Richard}M. Richard and X. Aim\'{e} and M.O. Krebs and J.
Charlet.  Enrich classifications in psychiatry with textual
data: an ontology for psychiatry including social concepts.  \textit{Studies in {H}ealth {T}echnology and {I}nformatics}, Vol. 210, pp. 221--223, 2015.
\end{thebibliography}


%-------------------------------------------------------
%\newpage
\abstracttitle{The Human Behaviour Project}
\abstractauthor[ Dietrich Albert ]{Dietrich Albert (TU Graz, AT)}
\license
%\abstractref[]{} % [] - URL, {} - reference description (a la thebibliography)
%\abstractrefurl{} % preferrably DOI-based URL
Understanding, predicting and modifying human behaviour of individuals and
groups in artificial and natural environments belong to the greatest challenges
facing 21st century basic \& applied sciences and research \& development
(R\&D). Rising to these challenges, we can (1) gain profound insights into human
behaviour and its underlying structures in all aspects, develop new methods for
behavioural diagnosing and predicting as well as new treatments for behavioural
changes and preventions, and (2) build revolutionary new computing technologies.
For the first time, modern information and communication technologies (ICT)
enable of tackling these goals. Thus, the project aims to achieve a multi-modal,
integrated understanding of behavioural structures and functioning through the
development and use of ICT. 

By bringing together
\begin{itemize}
\item machine readable theoretical and empirical content,
\item modern wireless sensors and manipulanda technology,
\item big behavioural data (offline and online),
\item techniques for dynamic representations of contexts and environments,
\item open and adaptive database methodology,
\item social media approaches,
\item technologies for machine learning and big data analytics,
\item semantic technologies etc.
\end{itemize}
totally new technologies for
\begin{itemize}
\item scalable, process-oriented, complex real time simulations
\end{itemize}
will be developed in
\begin{itemize}
\item strong co-operation of the behavioural sciences and the computer sciences.
\end{itemize}
Inherent
\begin{itemize}
\item process-oriented evaluative,
\item ethical components, and
\item gender aspects
\end{itemize}
will be implemented.

These technologies will realise simulations of individual and group behaviour in
different contexts and environments, large-scale collaboration and data sharing,
federated analysis of behavioural data, and the development of complex
integrated computing systems. Through the projects ICT platforms, scientists,
stakeholders, and engineers will be able to perform diverse experiments and
share knowledge with a common goal of unlocking human behaviour. With an
unprecedented cross-disciplinary scope, the project will integrate and stimulate
behavioural science, computing, and social science, will unify theory and
practice, and benefit the global scientific community dealing with humans. The
development and use of ICT will pave the way for the project's ultimate goal,
the simulation of human behaviour in terms of both individuals and groups in
artificial and real settings. 


%-------------------------------------------------------
%\newpage
\abstracttitle{Data Archiving and Sharing Confidential Data}
\abstractauthor[ George Alter ]{George Alter (University of Michigan – Ann Arbor, US)}
\license
%\abstractref[]{} % [] - URL, {} - reference description (a la thebibliography)
%\abstractrefurl{} % preferrably DOI-based URL
My presentation outlined the certification of ``trusted digital repositories'' and a framework for 
sharing confidential data.  Trusted digital repositories are expected to make data discoverable, 
meaningful, usable, trustworthy, and persistent.  This means that repositories must have 
procedures to document and preserve data and policies to sustain their institutional viability.   
``Deductive disclosure'' refers to re-identifying subjects from a combination of their 
characteristics in a data set.  Data repositories use a range of measures that providing access 
to the research community while minimize the risk of disclosure.  Procedures for sharing 
confidential data can be characterized under the headings: safe data (anonymization), safe 
places (data enclaves), safe people (legal agreements), and safe outputs (vetting computed 
results).   Since these measures are intrusive and hinder researchers, the severity of the 
measures should be weighed against the disclosure risks for each data set. 
  


%-------------------------------------------------------
%\newpage
\abstracttitle{\#dsos requires a digital infrastructure}
\abstractauthor[ Bjoern Brembs ]{Bjoern Brembs (Universit\"{a}t Regensburg, DE)}
\license
\abstractref[ http://journal.frontiersin.org/article/10.3389/fnhum.2013.00291/full ]{Brembs B, Button K and Munafò M (2013) Deep impact: unintended consequences of journal rank. Front. Hum. Neurosci. 7:291. doi: 10.3389/fnhum.2013.00291}
\abstractrefurl{http://journal.frontiersin.org/article/10.3389/fnhum.2013.00291/full} % preferrably DOI-based URL
Access is only one of many functionalities that are badly broken in our
scientific infrastructure. Our literature would lose little of its functionality
if we carved it in stone, took pictures of it and put them online. Our data – if
it is made accessible at all – all too often rests in financially insecure
databases. And our scientific code is hardly available at all, with no
institutional infrastructure to speak of. If the vision of digital scholarship
that is open by default is to become a reality, we need to raise funds to build
the digital infrastructure supporting digital open scholarship. On the local
level, we have developed proofs-of-concept, demonstrating the time-saving
potential of such an infrastructure. On the international institutional level, I
argue that we need to use the funds currently wasted on subscriptions to
implement this infrastructure as soon as possible.

  
%-------------------------------------------------------
%\newpage
\abstracttitle{VIVO – Connect, Share, Discover.  An open source, semantic web software system and ontology for representing scholarly work}
\abstractauthor[ Mike Conlon ]{Mike Conlon (University of Florida, US)}
\license
%\abstractref[]{} % [] - URL, {} - reference description (a la thebibliography)
%\abstractrefurl{} % preferrably DOI-based URL

\todo{draft by Christoph; get Mike's approval}VIVO is an open source, semantic web application, ontology and community providing standard data and tools for representing scholarship and using data about scholarship.
VIVO integrates and reuses institutional data about an organization, its scholars, its grants and projects, its publications and scholarly works, as well as its teaching and engagement.
From this it can generate reports, portfolios and curricula vitae, as well as visualizations.
The knowledge aggregated in a VIVO installation can help to find experts, to analyze networks, and to answer ad hoc queries.
It is reusable because it is exported as 5-star Linked Open Data.
VIVO has been widely adopted by research institutions around the world, predominantly in the US.

%-------------------------------------------------------
%\newpage
\abstracttitle{Research Objects for improved sharing and reproducibility in Psychology and Behavioural Sciences}
\abstractauthor[ Oscar Corcho ]{Oscar Corcho (Technical University of Madrid, ES)}
\license
%\abstractref[]{} % [] - URL, {} - reference description (a la thebibliography)
\abstractrefurl{http://www.slideshare.net/ocorcho/research-objects-for-improved-sharing-and-reproducibility} % preferrably DOI-based URL
When a researcher is working on a specific experiment, no matter what his/her scientific 
discipline is, a large amount of entities are used during the research process. This includes 
papers that have been read, input datasets, scripts, pieces of code, spreadsheets, output data, 
etc. Some time later, when this researcher goes back to all this material to resume this work, 
or another researcher wants to make use of it for another piece of research, he/she will 
normally find it very difficult to find all the material that was used at the time of the original 
investigation, to understand the purpose of some of those scripts, etc.

Research Objects have been proposed in the literature as a mechanism to aggregate all that 
material, making it more easily discoverable, providing identifiers to all these elements, and 
including metadata to understand better all these elements. More information available at 
\url{http://www.researchobject.org/} and details of the Research Object Model at 
\cite{Belhajjame201516}

During this Dagstuhl meeting we have had the opportunity to understand the type of 
resources that should be included in the most common types of Research Objects in the areas 
of Psychology and Behavioural Sciences, so as to propose in the future a Research Object 
profile that can be used in this area.

\begin{thebibliography}{0}
\bibitem{Belhajjame201516}Khalid Belhajjame, Jun Zhao, Daniel Garijo, Matthew Gamble, 
Kristina Hettne, Raul Palma, Eleni Mina, Oscar Corcho, Jos{\'e} Manuel 
G{\'o}mez-P{\'e}rez, Sean Bechhofer, Graham Klyne, Carole Goble. Using a suite of ontologies for preserving workflow-centric research objects.  \textit{Web Semantics: Science, Services and Agents on the World Wide Web}.  Vol. 32, pp. 16–42, 2015.  \href{http://dx.doi.org/10.1016/j.websem.2015.01.003}{\nolinkurl{doi:10.1016/j.websem.2015.01.003}}.
\end{thebibliography}

  


%-------------------------------------------------------
%\newpage
\abstracttitle{Estimating the Reproducibility of Psychological Science}
\abstractauthor[ Susann Fiedler ]{Susann Fiedler (MPG – Bonn, DE)}
\license
\jointwork{Open Science Collaboration}
\abstractref{Open Science Collaboration, Estimating the Reproducibility of Psychological Science, Science, in press}
%\abstractrefurl{} % preferrably DOI-based URL
Reproducibility is a defining feature of science, but the extent to which it characterizes 
current research is unknown.  We conducted replications of 100 experimental and 
correlational studies published in three psychology journals using high-powered designs and 
original materials when available.  Replication effects (Mr = .197, SD = .257) were half the 
magnitude of original effects (Mr = .403, SD = .188), representing a substantial decline.  
Ninety-seven percent of original studies had significant results (p < .05).  Thirty-six percent 
of replications had significant results; 47\% of original effect sizes were in the 95\% confidence 
interval of the replication effect size; 39\% of effects were subjectively rated to have replicated 
the original result; and, if no bias in original results is assumed, combining original and 
replication results left 68\% with significant effects. Correlational tests suggest that replication 
success was better predicted by the strength of original evidence than by characteristics of 
the original and replication teams.

%-------------------------------------------------------
%\newpage
\abstracttitle{Goals of the Seminar on Digital Scholarship and Open Science in Psychology and the Behavioral Sciences}
\abstractauthor[ Alexander Garcia Castro ]{Alexander Garcia Castro (Technical University of Madrid, ES)}
\license
%\abstractref[]{} % [] - URL, {} - reference description (a la thebibliography)
%\abstractrefurl{} % preferrably DOI-based URL

\todo{draft by Christoph; get Alex' approval}The ``Digital Scholarship and Open Science in Psychology and Behavioral Sciences'' seminar has two specific goals, namely:
\begin{itemize}
\item To foster and initiate the discussion about open science and
  digital scholarship in psychology and the behavioral sciences,
  addressing specific issues such as data standards, interoperability,
  knowledge representation, ontologies, linked data
\item To identify
  useful experiences from other domains, requirements, issues to be
  addressed. More importantly, to define a common vision, a road map
  for this community to build cyber infrastructures in support of open
  science and digital scholarship.
\end{itemize}

%-------------------------------------------------------
%\newpage
\abstracttitle{`Don't Publish, Release' Revisited}
\abstractauthor[ Paul Groth ]{Paul Groth (Elsevier Labs – Amsterdam, NL)}
\license
%\abstractref[]{} % [] - URL, {} - reference description (a la thebibliography)
%\abstractrefurl{} % preferrably DOI-based URL
At the 2013 Beyond the PDF 2 conference, Prof. Carole Goble's presented the
notion that research communication should be more like software development. It
is now evident, that many of the components in that vision are now a reality. We
can iterate, test, debug, execute and build upon our science using similar
tools. For example, journals such as Cognition allow for the pre-registration of
materials. Experiments performed primarily in-silico can be tracked and
reproduced using virtual machines. Version systems such as GitHub can be used to
keep track of versions, histories and dependencies. Such systems allow for
forking, staring, branching, which can be used to derive credit. Data
repositories allow for experimental data to be stored and cited. Notebook
environments such as Jupyter enable even creative analysis to be shared and
reproduced. As we go forward, these components will become seamlessly
interconnected. Such interconnection will enable new transparency and visibility
of the process of science. This increased visibility requires science to develop
new norms, namely, a new stance of constructive criticism. The openness enabled
by these technologies demands that we recognize that there are bugs and they can
be fixed. We should embrace the inherent interaction of science. 


%-------------------------------------------------------
%\newpage
\abstracttitle{Open Science Lessons Learned at Mendeley}
\abstractauthor[ William Gunn ]{William Gunn (Mendeley Ltd. – London, GB)}
\license
%\abstractref[]{} % [] - URL, {} - reference description (a la thebibliography)
%\abstractrefurl{} % preferrably DOI-based URL
Open Science is a new word for a old practice. What we now call Open Science
used to just be called science. In the early days, science wasn't funded by
national agencies, of course, but there were societies of learned gentlemen who
used to meet to share their results, write letters back and forth, etc. How and
why did that change? As technology grew in importance to society, science
professionalized and the discussion of science also had to professionalize.
Scholarly societies began to turn to companies like Elsevier for their ability
to run a journal, manage the operations, coordinate the peer review, publishing,
distribution, and so on. It made a lot of sense, back when publishing to a
worldwide audience necessarily had to be a difficult and expensive endeavor, to
let companies derive profit in exchange for the hard work of editing, producing,
and distributing research reports. Then the internet came along. We're not
entirely sure what scientific publishing on the Web will look like in 20 years,
but we're pretty sure that it won't continue to look like it has for the past
100+ years.

At Mendeley we have learned some lessons about Open Science, and we are
continuously drawing from other successful examples of leveraging the Web. One
way to make publication of research on the Web more like publication of other
things on the Web is to make it open, indexed, shareable, and available in
multiple forms. Because the research article is not the work, it's the report of
the work. It is essentially an advertisement that you have done a certain amount
of work, but that which is contained in your publication is only a very lossy
compression of your work, and your work consists of data generated, software
created, methods developed, and the broader impacts on society you've had. The
obvious thing to do is to connect these articles, these reports of the work, to
the work itself. Another example of an innovation to come by leveraging the Web
for open science is dynamic views of the primary data, instead of just the 2D
representation that was generated by the authors at the point of publication.
Detailed protocols for generating the data, software and virtual environments
that render the data into the view chosen by the author or into another view,
facilitate reader understanding of the strengths and limitations of the data as
collected and improve reproducibility. At Mendeley, we will continue to innovate
in these ways through working on the Resource Identification Initiative, the
Reproducibility Initiative, and building integrations with ELN tools like
Hivebench so that the provenance of an experiment is captured along with the
final outcome, allowing the work to be placed in content and built upon.
 


%-------------------------------------------------------
%\newpage
\abstracttitle{The role of standards and ontologies in tackling reproducibility}
\abstractauthor[ Janna Hastings ]{Janna Hastings (European Bioinformatics Institute – Cambridge, GB)}
\license
%\abstractref[]{} % [] - URL, {} - reference description (a la thebibliography)
%\abstractrefurl{} % preferrably DOI-based URL
 is a multi-faceted challenge. Standards and ontologies play an important role
in many of these facets. Reproducibility is enhanced through the provision of
raw data in open access repositories such that the analyses leading to results
can be entirely reproduced by different researchers and the data can be reused
for different research questions. However, for raw data to be truly reusable, it
must be presented in an accessible format and annotated in a standardised
fashion using shared ontologies. A minimum amount of information about the way
in which the data was generated needs to be provided, as well as important
contextual information about the entities that were investigated and the purpose
of the study. One of the hardest problems in achieving the wide exchange and
sharing of well-annotated data is the sociological challenge of bringing
communities together in order to create, and proliferate the use of, good
standards. A standard is no use unless it is widely adopted by the full
community. Ensuring adoption and usage requires the development of good tools
supporting such usage and the tireless promotion of standards compliance across
the full community. 


%-------------------------------------------------------
%\newpage
\abstracttitle{Developing reproducible and reusable methods through research software engineering}
\abstractauthor[ Caroline Jay ]{Caroline Jay (University of Manchester, GB)}
\license
\jointwork{Jay, Caroline; Haines, Robert}
%\abstractref[]{} % [] - URL, {} - reference description (a la thebibliography)
%\abstractrefurl{} % preferrably DOI-based URL
Discussions around open science and digital scholarship often focus on the 
important topics of creating and applying data standards, and achieving a robust  
infrastructure to support research. That scientists will follow standard, or at 
least well-defined, methods and operating procedures is a given – a crucial first 
step in ensuring research is reproducible.

In reality, research methods in psychology are often far from standard; they 
continually and necessarily evolve to meet the challenges of understanding 
new forms of behaviour and interaction. In the domain of human-computer 
interaction (HCI), this is particularly true, as traditional paradigms for
investigating behaviour often cannot be directly applied to technology use.

In many psychological studies, software is firmly embedded in both the data 
collection and analysis processes: packages such as E-Prime and Tobii Studio are
popular tools for ensuring that reaction time and gaze data measurements are
taken reliably, and are straightforward to interpret. Both these software tools
are proprietary, however, and whilst this results in stability that is helpful from
the perspective of reproducibility, it is less useful from the perspective of open
science.

Truly achieving reproducibility is hard. The authors have been striving to ensure 
their science is open for several years, but issues such as incomplete raw data, data 
that cannot be published for ethical reasons, the use of proprietary software, hard-to-
decipher analysis scripts and unavailable experimental materials have all proved 
barriers to reaching this goal (see for example, \cite{Jay2013TR}).

At the University of Manchester, and in particular as part of the EPSRC-funded 
IDInteraction project (EP/M017133/1), we are trying to address these challenges, by 
developing open-source software methods that not only make it easy to 
reproduce experimental results, but are also suitable for reuse and extension. 
Underlying our new approach is one crucial factor: the recognition of software 
engineering as a first class citizen in the research process. By paying attention to the 
usability and sustainability of software during the 
experimental design process, rather than treating it as an afterthought (or ignoring 
it completely), we hope to develop tools and methods that can be used to demonstrate 
the reproducibility of our own work, and support further experiments in the future.

Convincing others of the utility of 'research software engineering', and embedding it 
in the mainstream of scientific activity, is likely to require a significant cultural shift. 
Both scientists and research funders must recognise that the additional resources 
necessary to support this activity are vital to the future of science. Excellence in 
software engineering practice is essential to developing reproducible and reusable 
methods; scientists (for now, at least), are unlikely to be able to achieve this alone. As 
such, people with a focus on software development are as vital to producing genuinely 
reproducible computational research as people with a focus on the science itself. 

\begin{thebibliography}{0}
\bibitem{Jay2013TR} C.~Jay, A.~Brown, S.~Harper.
\textsl{Predicting whether users view dynamic content on the world wide web}.
{\em ACM TOCHI}, 20(2), 2013.
\end{thebibliography}
 


%-------------------------------------------------------
%\newpage
\abstracttitle{Open science, mega analyses and problems in understanding the genetics of psychiatric disorders}
\abstractauthor[ Iris-Tatjana Kolassa ]{Iris-Tatjana Kolassa (Universit\"{a}t Ulm, DE)}
\license
%\abstractref[]{} % [] - URL, {} - reference description (a la thebibliography)
%\abstractrefurl{} % preferrably DOI-based URL
For a better understanding of the etiology of psychiatric disorders and in order
to develop new medication and successful treatments we need to combine data
originating from both clinical psychology and genetics, and combine studies from
around the world to increase sample sizes. An infrastructure that allows easy
data sharing and exchange of knowledge will be highly beneficial for this purpose. 
So-called `mega analyses' combine participant-level data from multiple different
original studies to reach sample sizes of up to tens of thousands of subjects
(in contrast to traditional `meta analyses', which combine summary results and
parameter estimates on aggregate levels). First such mega analyses have already
been conducted; however, their results have been disappointing so far. A recent
mega analysis of the Major Depressive Disorder Working Group of the Psychiatric
GWAS Consortium concluded that even a sample of 18.759 independent and unrelated
subjects with and without major depressive disorder is still underpowered to
detect genetic effects typical for complex traits. 
Besides sample size, one reason why mega-analyses have failed might be that they
do not consider gene x environment interactions, which are crucial if one wants
to understand psychiatric diseases. One example which demonstrates this
particularly well is posttraumatic stress disorder (PTSD) – a disorder that
requires experiencing a traumatic event and thus is an example of an inherent
gene x environment interaction. In experiencing a traumatic event, a fear
network is built up that contains sensations, emotions, cognitions and
interoceptive experiences associated with the traumatic situation. With
increasing number of traumatic event types experienced, the fear network
increases in size, and specific triggers can reactivate multiple traumatic
events experienced. With increasing traumatic load, the lifetime prevalence of
PTSD reaches 100\%, the symptom severity increases and the probability of
spontaneous remission decreases. However, genetic factors interact with trauma:
Some studies suggest that in the case of low trauma load, genetic factors might
play an important role, while in the case of extremely high traumatic load, the
environmental factor is more influential than the genetic constitution of an
individual. One important problem of all current studies on the role of genes in
the etiology, symptomatology and treatment of PTSD is that it is not easy to
quantify the environmental factor traumatic load, in particular across various
populations and studies. However, initial evidence shows that it needs to be
considered not only in the etiology of PTSD, but also in studies assessing the
dependency of treatment effects on genetic factors. Furthermore, the gene x
environment equation needs to be broadened, e.g. epigenetic modifications need
to be considered when assessing the effects of genotypes, and genetic pathway
analyses might be more helpful than single candidate gene association studies. 
Open access genetic data would be helpful for combining data of various studies,
for reanalyzing previous studies given new knowledge gained, and finally to spot
mistakes in statistical analyses by the scientific community, as the statistics
of gene x environment interactions is complex, and frequently errors occur in
how factors that might influence the dependent variable are controlled for,
e.g., if the groups differ significantly in this covariate (see~\cite{MillerChapman}).  Misunderstanding analysis of covariance, Journal of Abnormal
Psychology). However, while there are many reasons why open data would be highly
beneficial for this field of research, the protection of individual data is a
particularly sensitive topic when studying traumatic event types experienced in
highly sensitive populations (e.g. victims of wars, genocide or other
atrocities) as well as when analyzing not only single nucleotide polymorphisms
but also whole genome and epigenome data. 

\begin{thebibliography}{0}
\bibitem{MillerChapman}Miller, G.A., \& Chapman, J. P. (2001). Misunderstanding analysis of covariance.  \textit{Journal of Abnormal Psychology}, 110, 40–48.
\end{thebibliography}
  


%-------------------------------------------------------
%\newpage
\abstracttitle{Advancing Psychology and Behavioral Sciences in Brazil and World Wide}
\abstractauthor[ Silvia Koller ]{Silvia Koller (Federal University of Rio Grande do Sul, BR)}
\license
\abstractref[ https://scholar.google.com.br/citations?user=KF11ZRkAAAAJ\&hl=pt-BR\&oi=ao ]{Full Professor and Chair of the Center for Psychological Studies of At-Risk Populations in the Department of Psychology at the UFRGS. Scientific Chair of Brazilian Virtual Library in Psychology \href{http://www.vs-psi.org.br}}
\abstractrefurl{https://scholar.google.com.br/citations?user=KF11ZRkAAAAJ\&hl=pt-BR\&oi=ao} % preferrably DOI-based URL
The seminar ``Digital Scholarship and Open Science in Psychology'' and 
Behavioural Sciences took place in the week of 20th to July 25th 2015.

The multi and inter disciplinary workshop was attended by scientists of 
psychology, behavior analysis, computer, biologists and biomedical sciences. 
Principles such as accessibility, sharing, interoperability and the possibility of 
multiple uses of the knowledge produced in several areas were the basis for 
discussions during the whole week. It was emphasized issues related to 
ontology of information systems, open access knowledge and data sharing in 
science.

The breakthrough perspective of knowledge is intrinsically linked to the 
possibility of opening and systematic and ongoing sharing of related objects to 
research, beyond the one that just occur when of the publication of scientific 
articles.

There was continuing emphasis on the need for fluidity and systematic opening 
of knowledge generated and tested, so that the science of behavior and 
psychology effectively may produce and improve the quality of life of human 
beings through knowledge. To get to such a possibility is needed clear 
description, systematic, standardized and rigorous terms, methods and 
procedures, beyond just data and results.

In Brazil, SciELO and BVS Psychology are examples of broad dissemination of 
science in these areas. New perspectives, such as the creation of the Science 
Data Repository DadoPsi – \url{http://dadopsi.bvs-psi.org.br}, further enable the 
advancement of Psychology and Behavioral Sciences in Brazil.

Moreover, it is very well received and accepted among scientists participating 
in the seminar, the fact that Brazil favors open access to the content of their 
magazines through tools, such as Scielo and Pepsic.

All kinds of open science, either through new open tools, new forms of 
standardization of terms and methods based on well organized ontologies and 
metadata, will contribute to the advancement of the area. 


%-------------------------------------------------------
%\newpage
\abstracttitle{Scholarly Communication and Semantic Publishing: Technical Challenges, and Recent Applications to Social Sciences}
\abstractauthor[ Christoph Lange ]{Christoph Lange (Universit\"{a}t Bonn, DE)}
\license
%\abstractref[]{} % [] - URL, {} - reference description (a la thebibliography)
%\abstractrefurl{} % preferrably DOI-based URL
This contribution presents an overview on current \emph{technical} challenges to
digital scholarship and open science (DSOS), and to visions for overcoming them
the near future.
With a focus on technology, this overview is largely domain-independent;
however, it gives some specific insight into the domain of social science.

\paragraph*{Overview}

The overarching goal of my DSOS research is to enable scholars to share
knowledge in a FAIR (findable, accessible, interoperable, reusable\footnote{%
This notion of FAIRness has originally been introduced for research data by the
FORCE11 initiative on the Future of Research Communication and e-Scholarship;
see their guiding principles at \url{https://www.force11.org/node/6062}.%
}) way.
The key assumption underlying my research agenda is that FAIR sharing of
scholarly knowledge is possible with information and communication technology
that supports the complete process of research and scholarly communication
without ``media disruptions''\footnote{%
The term ``media disruption'' is my free translation of the German term
``Medienbruch'', which refers to a point in information processing where the
carrier medium of the information changes.
This change typically results in loss of information, dropping information
quality, or as least inefficiency.%
} between its individual steps.
My proposed solution is to employ \emph{linked data} technology for preserving
information created by scientists in experiments, by authors while writing, and
by reviewers while commenting on a paper, in an explicit way to enable
intelligent services to act on it – and to provide data-driven services to
readers, authors and reviewers inside the familiar environment of a paper.

In the following, I point out media disruptions between the tools that so far
support the scientific process.
I argue why linked data has the potential to overcome these disruptions.
I present first results and the further agendas of our ongoing projects in this
field, covering the perspective of collaborative work environments as well as
the publication of (meta)data that enable services.
Finally, this extended abstract makes the case for combining both strands of
research to support the idea of open access, which should not only be seen from
a legal perspective, with a sustainable technical foundation.

\paragraph*{Problem Statement}

The increasingly collaborative scientific process, from a project plan to the
design of an experiment, to collecting data, to interpreting them and writing
down that interpretation in a paper, to submitting that paper for peer review,
to publishing an accepted paper, to, finally, its consumption by readers who
find, read and cite it, is insufficiently supported by contemporary information
systems.
They support every \emph{individual} step, but media disruptions between steps
cause inefficiency or even loss of information.
Examples include:
\begin{itemize}
\item Word processors lack direct access to data.
\item There is no assistant that would automatically recommend authors where to
submit their paper (i.e.\ to a high-profile event whose topic the paper matches
and where the paper has a realistic chance to be accepted).
\item Reviewers do not provide feedback in the same environment in which authors
will be revising their papers.
\item Open access web publishing is restricted to document formats designed for
paper printing but neglecting the Web's accessibility and interactivity potential.
\item Readers, seeing a single, frozen view of the underlying data in a paper,
are unable to access the full extent and the further dimensions of the data.
\item Information that helps to assess the quality of a scientific publication,
such as the peer reviews it received, the history of the venue (conference or
journal) in which it has been published, and information on the context in which
it has been cited, are scattered over different places, or not even available in
a machine-comprehensible format.
\end{itemize}

\paragraph*{The Potential of Linked Data Technology}

My research is based on the assumption that web technology, in particular
\emph{semantic} web and linked data technology, can address these problems, for
two reasons:
\begin{enumerate}
\item its potential to integrate heterogeneous systems and heterogeneous data:
Isolated solutions, such as tools for publishing data on the Web for easy
retrieval and visualisation, exist in preliminary manifestations in the social
sciences and other domains, but have not been integrated into tools for writing,
reviewing and publishing articles.
\item its approach of making the structure and semantics of data and documents
explicit to machines: document browsers that use articles as interactive
interfaces to related information on the Web, tools that make knowledge FAIR and
even remixable, as well as tools that assist writers in making their texts
machine-comprehensible with little additional effort, have been deployed
successfully in the life sciences and other fields.
\end{enumerate}

My research aims at transferring these ideas to the social sciences and beyond
by integrating existing data and publication management services into a
web-based collaborative writing environment that publishers can set up to
support all types of end users throughout the publication process: authors,
reviewers and readers.
To ensure acceptance by decision makers and end users, specifically
non-technical users, and to take advantage of existing solutions, even if
isolated, new collaboration environments should be compatible with the existing
solutions.
If a seamless integration is not feasible, compatibility should at least be
established by import/export interfaces or by well-defined migration paths.
Such flexible levels of integration are easiest to achieve based on free, open,
well-documented and extensible systems that already do part of the job.
For example, the collaborative document editor Fidus
Writer\footnote{\url{http://www.fiduswriter.org}} and the Open Journal Systems
submission and review management system\footnote{\url{https://pkp.sfu.ca/ojs/}}
provide a stable technical foundation for such integration efforts.

\paragraph*{Ongoing Efforts and First Results}

We are working on a \emph{collaborative writing environment} as outlined above
in the concrete setting of the `Opening Scholarly Communication in the Social
Sciences' (OSCOSS) project, which will run from autumn 2015 to autumn 2017 and
involves, besides the University of Bonn, the GESIS social science
institute\footnote{\url{http://www.gesis.org}} as an application partner.
In this project, we aim at securing user acceptance primarily by respecting the
characteristics of the traditional processes social scientists are used to: web
publications must have the same high-quality layout as print publications, and
information must remain citable by stable page numbers.
To ensure we meet these requirements, we will work closely with the publishers
of `methods, data, analyses' (mda) and `Historical Social Research' (HSR), two
international peer reviewed open access journals published by GESIS, and build
early demonstrators for usability evaluation.
The OSCOSS system will initially provide readers, authors and reviewers with an
alternative, thus having the potential to gain wider acceptance and gradually
replace the old, incoherent publication process of the participating journals.

Secondly, \emph{data and metadata} of which scientists could take advantage,
while doing their research and writing about it, is increasingly available on
the Web; however, there are two key limitations:
\begin{enumerate}
\item The datasets provide insufficient details and are often merely
superficially machine-comprehensible because valuable information is lost before
or during their publication: for example, …
  \begin{itemize}
  \item there are dataset registries, such as da|ra for the social
sciences\footnote{Registration agency for social and economic data; see
\url{http://www.da-ra.de}}, that make datasets retrievable and citable by
publishing metadata about them, but so far they do not effectively enable the
maintainers of datasets to publish the \emph{content} of their datasets in a
machine-comprehensible and thus FAIR way.
  \item Also, publication databases hardly allow for assessing the excellence of
a publication, researcher or venue in a way more comprehensive than counting
citations, as further contextual information is not published (e.g., acceptance
rates) or not yet easy to exploit (e.g., information on the structure and
dynamics of research communities or on the context in which sources are cited).
  \end{itemize}
\item Comprehensive services such as a conference/journal recommender assistant
would have to utilise data and metadata from multiple sources; however, there is
so far little interlinking between such data sources.
\end{enumerate}
Our work in the context of the OpenAIRE2020 European project (OpenAIRE = Open
Access Infrastructure for Research in
Europe\footnote{\url{http://www.openaire.eu}}) running from 2015 to 2018, where
the University of Bonn is leading the Linked Open Data (LOD) activities, on the
so far two editions of the Semantic Publishing
Challenge\footnote{\url{https://github.com/ceurws/lod/wiki/SemPub2015}}, and my
work as technical editor of the CEUR-WS.org open access publication service for
computer science addresses these problems.
In OpenAIRE2020, we are concerned with publishing metadata about all EU-funded
research projects, their results (publications and datasets, soon also software)
and their participating organisations and persons as
LOD\footnote{\url{http://lod.openaire.eu}}, and in a second step to interlink
them with related datasets, or to enrich them with information from other open
datasets with which they cannot be interlinked.
The 2014 and 2015 Semantic Publishing Challenges have addressed the problem of
extracting information from publications and proceedings that would help to
better assess their quality, e.g., information on the history of event series
and on citation contexts.
These challenges work on the data of CEUR-WS.org, paving the path towards
publishing them as LOD.
The other part of my work at CEUR-WS.org is, similarly to the work planned in
the OSCOSS project mentioned above, concerned with reducing the loss of
information in the publication process by avoiding media disruptions (e.g., by
enabling direct generation of high-quality proceedings volumes from the
EasyChair submission system used widely in computer science\footnote{%
See the ceur-make tool at \url{https://github.com/ceurws/ceur-make}}.%
), and with lowering the barrier to publishing proceedings and papers in a
machine-comprehensible and thus FAIR way for authors and chairs who are not
really ``non-technical'' (as we are in computer science), but, as experience
shows, busy or lazy and therefore not willing to waste time.

\paragraph*{An Integrated Technical Foundation for Open Access}

In summary, both strands of research outlined above – integrated collaboration
environments for readers, authors, reviewers, and the provision of
higher-quality research data and metadata – are expected to yield promising
results, partly supported by project funding until 2017/2018.
However, their \emph{synthesis} calls for a new project, which promises the
following two benefits:
\begin{enumerate}
\item data-driven services for readers, authors and reviewers, all accessible
from inside the familiar environment of a paper without loss of information
caused by media disruptions and without loss of efficiency caused by switching
tasks.
  Such services include:
  \begin{itemize}
  \item reusing and remixing data while reading/writing/reviewing.
    Making the links between tables and figures and their underlying datasets
explicit enables interactive document players to support users in exploring
different scenarios beyond the restricted scope chosen by the author.
  \item recommendations of citations based on the local context of the author's
current position in the document, and recommendations of publication venues
based on the structure and full text of a paper.
  \end{itemize}
\item an advanced collaboration environment that generates, during the normal
flow of interacting with it, and at no extra cost, machine-comprehensible
metadata that give others
– open access repository maintainers as well as immediate readers – FAIR access
to the scientific results produced inside the environment.
\end{enumerate}

Same as ``open data'' in the narrow sense is merely a legal framework for making
data reusable, while \emph{linked} data technology enables its practical
realisation\footnote{This is practically explained at
\url{http://5stardata.info}.}, the tight integration of research data and
metadata with collaborative writing and reviewing environments will serve as a
technical companion to the legal concept of \emph{open access}.
It will make journals more ``open'' (in terms of FAIRness) that are, legally,
open access already, and it has the potential to serve as an incentive for
turning ``closed'' journals into open access ones.

%-------------------------------------------------------
%\newpage
\abstracttitle{Defining the Scholarly Commons: Are We There Yet: Summary of my presentation and some thoughts on the workshop}
\abstractauthor[ Maryann Martone ]{Maryann Martone (UC – San Diego, US)}
\license
%\abstractref[]{} % [] - URL, {} - reference description (a la thebibliography)
%\abstractrefurl{} % preferrably DOI-based URL
In this presentation, I went through experiences in the neurosciences with
aggregating and searching across large amounts of data, based on our experiences
in designing and operating the Neuroscience Information Framework (NIF).  I
particularly focused on the importance of ontologies for providing a conceptual
backbone for search and organization of data across many different scales and
disciplines.  Because there is no single data type or technique that defines
neuroscience, without the conceptual underpinnings, there is not way to bring
together the different types of information, nor for searching across the
hundreds of millions of records contained in thousands of databases and data sets.  

The Neuroscience Information Framework, and its sister project SciCrunch, also
provide a practical data set for examining the current resource landscape.  NIF
has been cataloging and tracking research resources (data, tools, materials) for
over 8 years.  We see that funders are very willing to set up these resources,
but a remarkable number of them grow stale or disappear because of lack of
support.  

Although in any endeavor, it is common for a large number of initiatives to be
started and only some to take off, given the funding difficulties currently
facing biomedicine, this `launch and languish' model is not very cost effective.
 I talked a bit out SciCrunch, our configurable data portal technology, which
allows communities to create their own portals, customized to their needs and
branded with their own identity.  However, SciCrunch portals are connected on
the back end by a shared data infrastructure, so that any data added or improved
propagates throughout the network automatically.

I also discussed the power of web-based annotation as a means to add a
connecting and interactive knowledge layer on top of our scholarly output. 
Hypothesis is a non-profit that has developed the capability of annotating the
web.  Anyone by installing a plug in can highlight text on a web page or PDF and
add an annotation.  It is an open tool being engineered for an emerging W3C
standard for web annotation.

With Hypothesis, we can open up new information channels across static
publications, and add critical and currently missing knowledge about things like
reproducibility.  Alec Smecher of Open Journal Systems showed how Hypothesis was
integrated into the OJS platform.

I concluded by sharing some practical lessons that we've learned about open
science.  One of the most important is that people are important to this
endeavor:  data sharing and open science doesn't just magically happen.  It
needs champions and the societies need to support it.  Data resources become
interesting when there is a lot of data, so means to match requirements for data
sharing to our current incentive system is key.  

Finally, I believe that the period of letting a thousand flowers bloom in
creating these resources is past.  We have to invest in existing infrastructure,
e.g., institutional repositories and community repositories, to make them
better.  Our current trend is to look at them, find fault with them, and then
start again.  But this practice leads to `partially built cars'.  What do I mean
by that?  Think of current research infrastructures as cars.  If our current
funding model built fully functional cars, then some would win and some would
lose but we'd still have cars to drive.  If our current system didn't build cars
but car parts, then someone could take these interoperable pieces and build a
car that works.  But what we currently do is build partially built,
non-interoperable cars.  So we may be able to limp along in one or two, but none
can fully thrive.

We think that it is time for the community to start to come together around the
idea of the Scholarly Commons-that is, the Set of protocols, principles, best
practices, API's and standards that govern flow of scholarly research object. 
The ultimate goal is to make research objects (the sum total of research output)
FAIR:  Findable, accessible, interoperable, reusable.  Through organizations
like FORCE11 (Future of Research Communications and e-Scholarship), we are
getting closer to be able to articulate what is required for research
communities to become part of the commons.  

Thoughts:  This workshop was an extremely valuable discussion forum for bringing
several of the concepts in my talk into clearer focus.  The exercise where we
designed a future system and then realistically assessed where we stood led to
the concept of making communities `e Science ready' as opposed to overpraising
what we can do today.  What became clearer, and what I have used in talks since
then is that there are some things that communities need to make the transition
from non-digital to digital.  Because these researchers in our current reward
system are not likely to accrue significant benefits in the beginnings, the
`asks' cannot be overly onerous.  

So what is required:

\begin{enumerate}
\item People, concepts, instruments and materials need to enter the eScience
world with a persistent identifier attached.  Efforts like ORCID and the
Resource Identification Initiative are making headway and should be supported. 
If a community doesn't have an open ontology or controlled vocabulary, they need
to support the creation of one.  If they can't make their instruments, e.g.,
questionnaires, eScience enabled (i.e., unique ID, network accessible), then
they need appropriate tools to do so.  The latter tools should help their
science by making it easier for them to create what they need and not hinder it.
\item Data needs to be made potentially accessible and recoverable.  The
strongest argument for data sharing right now is the transparency argument, that
is, all data that was produced in the course of the study needs to be made
available and potentially recoverable in the future.  That means having the data
hosted in an appropriate repository and having at least minimal standardized
metadata.  We have the means to do both and it doesn't take a lot of the
researchers time to work with curators to achieve this.  Proper norms about what
should be shared publicly (or not) and when need to be developed in conjunction
with the community, but if data are properly deposited and stewarded, then the
data can be shared when these agreements are reached.  If we don't make the data
potentially recoverable, then we lose it for all time.
\end{enumerate}


%-------------------------------------------------------
%\newpage
\abstracttitle{Cognitive ontologies, data sharing, and reproducibility}
\abstractauthor[ Russell Poldrack ]{Russell Poldrack (Stanford University, US)}
\license
%\abstractref[]{} % [] - URL, {} - reference description (a la thebibliography)
%\abstractrefurl{} % preferrably DOI-based URL
In my talk, I outline the need for formal ontologies to describe cognitive 
processes, and provide an overview of the Cognitive Atlas project, which aims to 
develop such an ontology.  I describe the structure of the Cognitive Atlas, 
focusing particularly on the different classes of entities (tasks and concepts)
that 
are represented in the knowledge base.  The Cognitive Atlas has been used to 
annotate the OpenfMRI database of neuroimaging data, and this annotation has 
been used to support ontology-driven data mining.  I also outline ongoing work in 
our group on reproducibility in the context of neuroimaging, discussing the 
threats to reproducibility that are inherent in current practices and describing
the 
work of the Stanford Center for Reproducible Neuroscience, which is developing a 
new resource to allow researchers to better quantify the reproducibility of their 
findings.
 


%-------------------------------------------------------
%\newpage
\abstracttitle{Open Journal Systems: Introduction, Preview, and Community}
\abstractauthor[ Alec Smecher ]{Alec Smecher (Simon Fraser University – Burnaby, CA)}
\license
%\abstractref[]{} % [] - URL, {} - reference description (a la thebibliography)
%\abstractrefurl{} % preferrably DOI-based URL
Open Journal Systems is a widely used open source web application providing a
complete journal publishing workflow, emphasizing (but not exclusive to) Open
Access publishing by automating the time-consuming and expensive workflow
process. It is written and maintained by the Public Knowledge Project (PKP),
\url{http://pkp.sfu.ca}, with contributions of code, translations, etc. from a
diverse community of contributors.

In the 13 years since OJS 1.0 was released, it has grown from a proof of concept
into a mature piece of infrastructure helping to facilitate the publishing of
many thousands of journals in over 30 languages.

More recently, PKP has successfully introduced a new platform for managing
scholarly monographs and edited volumes. Open Monograph Press (OMP) was also
used as an opportunity to pioneer a rewrite of aspects of OJS that were aging or
in need of updating to keep pace with new scholarly publishing trends.

In August 2015 PKP will unveil a beta release of OJS 3.0, including numerous
technical and workflow improvements.

At the 2015 Dagstuhl conference on Digital Scholarship and Open Science in
Psychology and the Behavioral Sciences, crossover discussions between open
science and open source have frequently arisen, both in terms of the importance
of Free and Open Source Software to scientific replicability, and of the
potential for tools and workflows from the open source software development
community to be studied and potentially introduced into the future practices of
psychology research.


%-------------------------------------------------------
%\newpage
\abstracttitle{Principles, Programs and Pilots for Open Science and Digital Scholarship at Elsevier}
\abstractauthor[ Daniel Staemmler ]{Daniel Staemmler (Elsevier Publishing – Berlin, DE)}
\license
%\abstractref[]{} % [] - URL, {} - reference description (a la thebibliography)
%\abstractrefurl{} % preferrably DOI-based URL
Elsevier supports the storing, sharing, discovering, and using of research data. Elsevier established a research data policy based on the STM Brussels Declaration 2007\footnote{\url{http://www.stm-assoc.org/2007_11_01_Brussels_Declaration.pdf}} that “Raw research data should be made freely available to all researchers wherever possible” to help researchers to store, share, discover and use data.

\noindent The following principles underpin Elsevier's data policy\footnote{\url{https://www.elsevier.com/about/company-information/policies/research-data}}:
\begin{itemize}
\item Research data should be made available free of charge to all researchers wherever possible and with minimal reuse restrictions.
\item Researchers invest substantially to create and interpret data and others such as data archives, publishers, funders and institutions further add value and/or incur significant cost. In all such cases these contributions need to be recognized and valued.
\item Expectations and practices around research data vary between disciplines and need to be taken into account.
\item Platforms, publications, tools and services can enhance data by improving their discoverability, use, reuse, and citation.
\item Standard identifiers, vocabularies, taxonomies, ontologies and entity resources enhance the discovery, management and use of data.
\end{itemize}
\noindent The following programs and pilots have been initiated:
\begin{enumerate}
\item \emph{Data-linking program}\footnote{\url{http://www.elsevier.com/databaselinkin}}\\
Elsevier has an extensive program with 40+ leading domain-specific data repositories to interlink articles and data on ScienceDirect. This reciprocal linking aims to expand the availability of research data and improve the researcher workflow. Researchers – whether in the role of author or reader – benefit from both the increased discoverability of the data sets and seeing the data sets in the direct context of the research article. Linking through in-article accession numbers, data DOIs, or data banners are two examples on how this is being accomplished.
\item \emph{Mendeley Data}\\
Allows researchers to store their research data online, so it can be cited and shares as well as securely saved in an online repository. DOIs and versioning of datasets, in compliance in Force11 standards, ensure that data citations are always valid. Mendeley Data is currently in beta phase.
\item \emph{In-article data visualization}
  \begin{enumerate}
  \item iPlots\footnote{\url{https://www.elsevier.com/books-and-journals/content-innovation/iplots}} – Displaying plot data in CSV format delivered by the author as supplementary material. Allows to access, explore, and download data behind plots.
  \item 3D visualization tool\footnote{\url{http://www.elsevier.com/connect/bringing-3d-visualization-to-online-research-articles}} – The goal is to enable Elsevier authors to showcase their 3D data, and to provide ScienceDirect users with a means to view and interact with these author-provided small to massive 3D datasets on a large number of devices with no additional plug-in required. These devices include smartphones, tablets, laptops and desktop computers.
  \end{enumerate}
\item \emph{Open data and data profile}\footnote{\url{https://www.elsevier.com/about/open-science/research-data/open-data}}\\ Increasing access to research data helps researchers to validate and build upon important discoveries and observations. With Elsevier's latest Open Data pilot we are providing authors with the opportunity to make their supplementary files with raw research data available open access on ScienceDirect.
\item \emph{Data micro-articles}\\
Data journals, and data sections in existing journals, enable authors to have their research data peer-reviewed and cited. It will also make sure readers can find, use and analyze the data hosted in external databases or submitted as supplementary data. Examples of recently launched data journals are Genomics Data and Data in Brief.
\item \emph{Standards bodies and working groups}
  \begin{enumerate}
  \item Joint Declaration of Data Citation Principles: best-practices to cite data in articles for better linking and credit
  \item Research Data Alliance \& ICSU World Data System: Tackling a broad range of interconnected issues around Data Publication (workflows, bibliometrics, cost recovery, services)
  \end{enumerate}
\item \emph{Lay Summaries}\\
Making scientific research results accessible to the public by posting for each published article in the journal “Burnout Research” a lay summary explaining the main research findings. Burnout Research is one of Elsevier's 270 open access titles and therefor freely available on the web (link to lay summaries).
\item \emph{STM Digest}\\
\footnote{\url{http://www.elsevier.com/social-sciences/economics-and-finance/early-career-researchers}} STM Digest features lay summaries of science papers with societal impact. It is a collection of summaries of original research papers with social impact or a focus on policy. These summaries have the potential to make research more accessible, improve engagement in science, and benefit wider society. The initiative is a collaboration between Elsevier's STM Journals group and the cloud-based research management and social collaboration platform, Mendeley.
\item \emph{Atlas}\footnote{\url{https://www.elsevier.com/atlas/home}}\\
With over 1,800 journals publishing articles from across science, technology and health, Elsevier's mission is to share some of the stories that matter. Each month Atlas showcases research that could significantly impact people's lives around the world or has already done so. Bringing wider attention to this research will hopefully go some way to ensuring its successful implementation. Each month selecting a single article to be awarded “The Atlas” is facilitated by the Advisory Board. The winning research is presented in a lay-friendly, story format alongside interviews, expert opinions, and multimedia to reach a wide global audience.
\end{enumerate}

%-------------------------------------------------------
%\newpage
\abstracttitle{Open data and the need for ontologies}
\abstractauthor[ Robert Stevens ]{Robert Stevens (University of Manchester, GB)}
\license
%\abstractref[]{} % [] - URL, {} - reference description (a la thebibliography)
%\abstractrefurl{} % preferrably DOI-based URL

This is an abstract for ``Digital Scholarship and Open Science in Psychology and
the Behavioural Sciences'', a Dagstuhl Perspectives Workshop (15302) held in the
week commencing 20 July 2015. The workshop brought together computer scientists,
computational biologists and people from the behavioural sciences. The workshop
explored eScience, data, data standards and ontologies in psychology and other
behavioural sciences. This abstract gives my view on the advent of eScience in
parts of biology and the role open data and metadata supplied by ontologies
played in this change.

There is a path that can be traced with the use of open data in the biological
domain and the rise in the use of ontologies for describing those data. Biology
has had open repositories for its nucleic acid and protein sequence data and
controlled vocabularies were used to describe those data. These sequence data
are core, ground truth in biology; all else comes from nucleic acids and, these
days, the environment. As whole genome sequences became available, different
organism communities found that the common vocabulary used to represent
sequences facilitated their comparison at that level, but a lack of a common
vocabulary for what was known about those sequences blocked the comparison of
the knowledge of those sequences. Thus we could tell that sequence A and
sequence B were very similar, but finding that the function, processes in which
they were involved and where they were to be found etc. was much more difficult,
especially for computers. Thus biologists created common vocabularies, delivered
by ontologies, for describing the knowledge held about sequences. This has
spread too many types of data and many types of biological phenomenon, from
genotype to phenotype and beyond, so that there is now a rich, common language
for describing what we know about biological entities of many types.

At roughly the same time was the advent of eScience. The availability of data
and tools open and available via the Web, together with sufficient network
infra-structure to use them, led to systems that co-ordinated distributed
resources to achieve some scientific goal, often in the form of workflows.  Open
tools, open data, open standards, open, common metadata all contribute to this
working, but it can be done in  stages; not all has to be perfect for something
to happen – just availability of data will help, irrespective of its metadata.
Open data will, however provoke the advent of common data and metadata
standards, as people wish to do more and do it more easily.

In summary, we can use the FAIR principles (Findable, Accessible, Interoperable
and re-usable) to chart this story. First we need data and tools to be
accessible and this means openness. Metadata, via ontologies, also have a   role
to play in this accessibility – do we know what those data are etc.? Metadata
has an obvious role in making tools and data findable – calling the same things
by the same term and knowing what those terms mean makes things findable. The
same argument works for interoperable tools and data.



%-------------------------------------------------------
%\newpage
\abstracttitle{Infrastructural Services for the Scientific Community provided by the American Psychological Association}
\abstractauthor[ Gary VandenBos ]{Gary VandenBos (American Psychological Association, US)}
\license
%\abstractref[]{} % [] - URL, {} - reference description (a la thebibliography)
%\abstractrefurl{} % preferrably DOI-based URL
My input to this workshop is based on my experience as the Publisher of the
American Psychological Association\footnote{\url{http://www.apa.org/}} in
Washington, DC, USA, and as the co-Editor of the Archives of Scientific
Psychology\footnote{\url{http://www.apa.org/pubs/journals/arc/}}, an open
methods, collaborative data sharing, open access journal.  I have designed
electronic knowledge dissemination products for the field of psychology since
1984, including moving the Psychological Abstracts from a print product to a
CD-based electronic product to
PsycINFO\footnote{\url{http://www.apa.org/pubs/databases/psycinfo/}}, a
streaming Internet product.  I also developed
PsycARTICLES\footnote{\url{http://www.apa.org/pubs/databases/psycarticles/}} (a
full-text journal article database),
PsycBOOKS\footnote{\url{http://www.apa.org/pubs/databases/psycbooks/}} (a
full-text book and book chapter database),
PsycTESTS\footnote{\url{http://www.apa.org/pubs/databases/psyctests/}} (a
measurement instrument database), and
PsycTHERAPY\footnote{\url{http://www.apa.org/pubs/databases/psyctherapy/}} (a
streaming video database of psychotherapy demonstrations).  I am the Editor of
the Publication Manual of the American Psychological
Association\footnote{\url{http://www.apastyle.org/manual/}}.  I have been an
advocate for data sharing since 1990, and have served on many governmental and
association task forces on data sharing – including the recent TOP Guidelines
developed by the Center for Open Science\footnote{\url{https://cos.io/top/}}.   


%-------------------------------------------------------
%\newpage
\abstracttitle{Hijacking ORCID}
\abstractauthor[ Hal Warren ]{Hal Warren (Vedatek Knowledge Systems, US)}
\license
%\abstractref[]{} % [] - URL, {} - reference description (a la thebibliography)
%\abstractrefurl{} % preferrably DOI-based URL
The Open Researcher and Contributor ID (ORCID) is a subset of an International 
Standard Name Identifier (ISNI), a 16 digit number that serves as a persistent 
identifier.  This persistence turns human data into individually known machine 
readable data that can remain until the end of our civilization.  The Internet was 
first the domain of scholars.  ORCID was created as a means to disambiguate 
works of scholarly authors.  It is time to broaden the audience for ORCID to 
everyone, taking advantage of persistence to join all our public personas into a 
single identifier, ORCID.  By using the ORCID record to connect my Uniform 
Resource Identifiers (URIs) such as my Facebook account, my Twitter account as 
well as all of my email addresses, each instance of me can serve as a legitimate 
identifier of me which can be verified against the ORCID record.  ORCID adds 
credibility and provenance to whom I am online by joining different silos of my 
data so that machines can better reason on it.

Scholarly publishers are positioned to take advantage of ORCID by adding 
advanced machine reasoning to better structure disambiguated data.  By 
assisting authors with the ORCID update process, needed infrastructure to 
support Research Object-based academic credit will emerge.  My annotations are 
automatically connected to me regardless of the channel in which they are 
created.  I become more complete.

By joining our health, financial, contribution and consumption data through 
ORCID, we create a trusted digital corpus with new capacity.  Vedatek Knowledge 
Systems is hijacking ORCID for ordinary citizens, to improve their quality of life 
through the use of new sensor data and to augment the growth of local 
community connections. 

%-------------------------------------------------------
%\newpage
\abstracttitle{PsychOpen – The European Open-Access Publishing Platform for Psychology}
\abstractauthor[ Erich Weichselgartner ]{Erich Weichselgartner (ZPID – Trier, DE)}
\license
%\abstractref[]{} % [] - URL, {} - reference description (a la thebibliography)
%\abstractrefurl{} % preferrably DOI-based URL

The European Psychology Publication Platform PsychOpen was created because extensive research in the European scientific community had clearly shown a demand for open access publishing in psychology. The reasons were manifold. For one, there were only a handful of quality controlled open access journals in psychology in 2011. Secondly, a survey from 493 participants from 24 countries had revealed six main concerns with traditional publishing in psychology: (1) Language, (2) review process, (3) manuscript handling, (4) impact (visibility), (5) permission barriers (accessibility) and (6) price barriers (cost). These issues are the concerns of non-native English speaking Europeans as they experienced in their home countries. PsychOpen was founded in 2013 on the conclusion that an open-access infrastructure would boost scientific and professional communication in European psychology, especially when Europe's language diversity and the lack of resources at the national level (e.g. in Eastern Europe) are taken into account. For the latter reason, to remove hurdles for developing countries and Eastern Europe, but also for strict separation of economic interests and quality control, PsychOpen is Gold Open Access without any author fees.

In order to accomplish its goals efficiently on a small budget, PsychOpen uses a mix of commercial and open source publishing software like PKP's Open Journal System and Inera's eXtyles. Two years after its start, PsychOpen publishes seven journals: Publication languages are English (> 80\%), Bulgarian (Non-Roman Script), Portuguese, Spanish and German. The scope is mostly research, one journal is devoted to professional topics. The publication type is traditional research articles. The average publishing time is four months. The publication schedule is continuous in one instance and discrete for the other six journals. Submissions per year and journal range from 25–150; rejection rates are 20\%–65\%.

All content is published according to the Creative Commons license CC-BY. Two third of PsychOpen authors have a European affiliation; the remaining one third come from North America, East Asia, South America, Africa and South Pacific (in this order). Usage is up by 44\% from 2014 to 2015 with approx. 50.000 article downloads in mid-2015. Amongst the challenges for PsychOpen that need further work is multilingualism, the interlinking of scholarly content (e.g., research articles with the corresponding research data), the integration of social media and semantic publishing. A new tool for the semantic enhancement for the Open Journal System facilitates the generation of RDF. Resulting self-describing documents for scientific literature in psychology will allow discovering connections amongst papers and concept-based queries. The lack of ontologies and of NLP tools in psychology, but also the poor data infrastructure are hurdles that need to be overcome.

\end{document}

% LocalWords:  Dagstuhl Christoph Weichselgartner biomedical BAM APA
% LocalWords:  APS Aime LIMICS INSERM Charlet Odile Aimé Informatics
% LocalWords:  CIM DSM OntoPsychia PDS ealth echnology nformatics NL
% LocalWords:  Graz ICT manipulanda dsos Bjoern Brembs Universit doi
% LocalWords:  Regensburg Munafò Neurosci VIVO Conlon Corcho Zhao HCI
% LocalWords:  Belhajjame Garijo Hettne Palma Eleni Mina Jos mez rez
% LocalWords:  Bechhofer Klyne Goble centric Susann Fiedler Groth EP
% LocalWords:  Elsevier Goble's Jupyter Mendeley Gunn ELN Hivebench
% LocalWords:  Bioinformatics Haines Behaviour behaviour behavioural
% LocalWords:  realise standardised Tobii IDInteraction recognise Ulm
%  LocalWords:  TOCHI Tatjana Kolassa GWAS underpowered posttraumatic
%  LocalWords:  PTSD Koller Grande Sul hl ao UFRGS th SciELO BVS mda
%  LocalWords:  DadoPsi Scielo Pepsic FAIRness Medienbruch Fidus HSR
%  LocalWords:  visualisation OSCOSS GESIS da ra utilise OpenAIRE WS
%  LocalWords:  CEUR organisations ceur realisation Martone UC NIF
%  LocalWords:  SciCrunch Smecher OJS API's accessable eScience ORCID
%  LocalWords:  Poldrack OpenfMRI neuroimaging Burnaby PKP OMP ISNI
%  LocalWords:  Staemmler SoftwareX MethodsX ordinated VandenBos ZPID
%  LocalWords:  PsycINFO PsycARTICLES PsycBOOKS PsycTESTS PsycTHERAPY
%  LocalWords:  Vedatek ZPID's PsychData PsychOpen PubPsych STM DOIs
%  LocalWords:  PsychAuthors Elsevier's ScienceDirect iPlots Genomics
%  LocalWords:  ICSU bibliometrics PKP's Inera's eXtyles Amongst NLP
%  LocalWords:  amongst
